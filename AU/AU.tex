\documentclass[10pt, letterpaper]{article}
        \usepackage[utf8]{inputenc}
        \usepackage[margin=1in]{geometry}
        \usepackage{fancyhdr}
        \usepackage{titling}
        \usepackage{enumitem}
        \usepackage{mathtools}
        \usepackage{amssymb}
        \usepackage{xfrac}
        \usepackage{booktabs}
        \usepackage{graphicx}
        \usepackage{wrapfig, blindtext}
        \usepackage{hyperref}
        \usepackage{enumerate}
        \usepackage{multicol}

        
        \setlength{\parindent}{0pt}

\title{Journal 2}
        \author{Sudhan Chitgopkar}
        \date{\today}
        
        % headers -- no need to change
        \pagestyle{fancy}
        \fancyhf{}
        \lhead{African Union}
        \chead{UGAMUNC XXVII}
        \rhead{\thedate}

\begin{document}

Greetings Delegates! \\

On behalf of the University of Georgia, welcome to
UGAMUNC XXVII and to the United Nations African Union! My name is
Cameron Kasper, and I am beyond excited to be chairing this body for the
upcoming conference. I am a first year Political Science and
International Affairs double major and a native of the Brunswick and St.
Simons Island area of Southeast Georgia. In my spare time, I enjoy being
with friends, watching Parks and Recreation, cheering on my favorite
football teams (Go Dawgs), and travelling. \\

This last year has been full of challenges, upheavals, trials,
tribulations, and seemingly never-ending chaos. However, I am glad we
are able to make a return to some normalcy through your participation in
this conference. Model UN has been a part of my life for over seven
years. In that time frame, I have forged some of my strongest bonds with
people across the world while competing at competitions across the
country. There are many people I have met that I still keep up with
today. That's exactly the kind of environment I would like to foster,
and I hope that if you gain nothing else from this experience that you
have fun and make friends along the way. \\

With that, I will note that the topics I have selected for this year's
simulation are intentionally broad and general. This background guide
should serve as a strong foundation on which to base your research, but
it is in no way a final say on the sub-topics you can explore within
these broad topics. I would prefer not to stifle your pursuance of
intellectual creativity with clear-cut topics on very specific issues
plaguing the African continent. It should also not serve as your only
source of information. I encourage you to use every resource at your
disposal to prepare, but I would also highly encourage you visit any
links or citations you see throughout the paper when beginning your
research. While I understand the provided topics are very broad, I
expect the exact opposite from your position papers and resolutions. A
high amount of research and preparation should go into your paper, and
you should come fully prepared to discuss with your peers the direction
you as a delegate would like to take, and I would also suggest that you
come prepared to talk about other issues your peers may want to discuss.
I'm looking for you to use your own intellectual creativity here, but
that should serve towards achieving a specific goal during committee
rather than providing a broad resolution that never actually achieves
anything. The real UN passes plenty of those empty resolutions on their
own. To summarize, be specific, build consensus, respectfully disagree
if necessary, but above all, do a quality job and be proud of your work. \\

As a delegate in this committee, it is expected that you also maintain a
high level of maturity and professionalism throughout the debate. I am
looking for cooperation that extends beyond just one or two people
leading every small group. Every voice is equally as important, and your
attitudes towards others should reflect that. At the beginning of the
conference, we will review parliamentary procedure and what is expected
of each delegate before we begin, but I urge every one of you to get a
solid foundation of UGAMUNC rules by reviewing them on our website. \\

I am looking forward to an exciting conference, and I cannot wait to
read your position papers. Please submit those to me at \texttt{\href{mailto:cjk82979@uga.edu}{cameron.kasper@uga.edu}} by
\textbf{Monday, February 1, 2021 at 11:59 P.M}. If you have any
questions, I am always here to help. Never hesitate to contact me via
email with any questions you have about MUN or the University of
Georgia, issues you are finding in your research, or just to update me
on your general preparation. See you all very soon! \\

Best,\\
Cameron Kasper, Chair of the African Union

\newpage
\tableofcontents
\newpage

\section{Committee Background}

The African Union was formally launched in July of 2002 in Durban, South
Africa. However, this was not the first time African countries had
agreed to cooperate in a governing body such as this. Many African
countries had already been cooperating through their participation in
the Organisation of African Unity (OAU), an organization founded in 1963
following growing calls for a united pan-African dream in the aftermath
of independence of most African colonial holdings. While the original
founding members of this cooperative body had been freed from colonial
rule, many areas of the African continent were much slower to achieve
independence, and some that were freed still struggled with apartheid
systems created by their former colonial rulers.\footnote{"About the
  African Union." About the African Union \textbar{} African Union.
  October 13, 2020. https://au.int/en/overview.} \\

We will first look at the establishment of the OAU to provide an outlook
for the necessity of the African Union when such a pan-African
cooperation organization already existed. The primary importance of the
OAU, as spelled out in their charter, was ``to eradicate all forms of
colonialism from Africa.''\footnote{Organization of African Unity.
  \emph{OAU Charter}. PDF.
  \href{https://au.int/sites/default/files/treaties/7759-file-oau_charter_1963.pdf}{\underline{https://au.int/sites/default/files/treaties/7759-file-oau\_charter\_1963.pdf}}}
We can assume that this verbiage referred both to the traditional
meaning of colonialism as well as the cessation of the apartheid system.
In 1980, Zimbabwe had become the last African nation to become
independent from a European power and therefore the last to overthrow
the shackles of the more traditional use of the word
colonialism.\footnote{Rfi. "Timeline: African Independence." RFI.
  February 19, 2010.
  https://www.rfi.fr/en/africa/20100216-timeline-african-independence.}
However, the problem of apartheid remained until the liberation of South
Africa in 1994.\footnote{U.S. Department of State.
  https://2001-2009.state.gov/r/pa/ho/time/pcw/98678.htm.} It was then
that the OAU had successfully achieved its goal to eradicate both forms
of colonialism from Africa. \\

While the OAU was viewed as a success in achieving its goal to eradicate
colonialism, it was largely silent in addressing other important issues
in post-colonial Africa. The OAU had a very strong commitment to
preserving the national sovereignty of its member states, leaving it
completely powerless to intervene in the many civil wars that raged in
multiple countries or to provide relief when a government was overthrown
and a military coup replaced it. The combination of the failure to
resolve issues such as these alongside the successful achievement of its
primary goal led ultimately to the breakup of the OAU, but it also
provided for the establishment of the more unifying African
Union.\footnote{"The Birth of the African Union." Exploring Africa.
  http://exploringafrica.matrix.msu.edu/the-birth-of-the-african-union/\#:\textasciitilde:text=In
  May, 1963 the leaders,relationships between independent African
  States.} \\

In its eighteen year history, the African Union has been extremely
active in the fight against corruption and seeking peace. While it has
been heavily criticized for its failure to issue strong condemnation of
human rights abuses, it has had a large impact on the African continent
as a whole through initiatives and commissions created to combat a wide
variety of issues plaguing member nations such as fighting the West Nile
Virus, aiding in the fight against AIDS, and controlling government
turmoil in the Sudan.\footnote{"Main Successes of the AU in Peace and
  Security, Challenges and Mitigation Measures in Place." Main Successes
  of the AU in Peace and Security, Challenges and Mitigation Measures in
  Place \textbar{} African Union. October 08, 2020.
  https://au.int/en/pressreleases/20170127/main-successes-au-peace-and-security-challenges-and-mitigation-measures-place.} \\

\newpage
\section{Topic A: Achieving the Sustainable Development Goals on the African Continent}

\subsection{Introduction}

In 2015, all United Nations Member States adopted the seventeen
Sustainable Development Goals (SDGs), also referred to as the Global
Goals, to act as a global call to advance human society by the year
2030. While most of the developed world has managed to overcome a lot of
these problems, the members of the African Union still struggle to
achieve in areas such as quality education, zero hunger, and gender
equality. The seventeen SDGs can be found here:
\href{https://sdgs.un.org/goals}{\underline{https://sdgs.un.org/goals}} \\

Despite the progress made in recent years, Africa remains plagued by
poverty and violence that is causing economic, political, and social
advancements to stagnate. The African Union Commission Chairman issued a
statement on May 25th of this year that noted that ``such progress
cannot conceal the sometimes-flagrant shortcomings and
delays.''\footnote{``Violence, Poverty Impede Africa's Progress: AU
  Leader,'' Anadolu Ajansı, accessed November 2, 2020,
  https://www.aa.com.tr/en/africa/violence-poverty-impede-africa-s-progress-au-leader/1852997.}
In other words, while Africa has maintained consistent growth towards
achieving these goals, it is oftentimes interrupted by issues that
require more immediate attention such as famines or wars. \\

The entire premise of the Sustainable Development Goals is to insure
that we are handing off a world to succeeding generations that is
equipped to tackle the challenges of that era, rather than consistently
being bogged down in the challenges we face today. With that, it is
important to note that Africa currently contains over 1.3 billion
people, accounting for 17\% of the total world population. Further, it
is also considered to be the fastest growing population center in the
world. More than half of global population growth between now and 2050
is expected to occur in Africa, and by 2050, the Sub-Saharan portion of
the African continent alone is expected to double in population size.
Other areas of the world appear to be stagnating or even declining due
to more older individuals than younger ones. In Africa, this is exactly
the opposite. This means that Africa has a large role to play in the
size and scope of the future population, even if family planning
assistance measures are implemented by governments. This leaves Africa
as one of the most important regions of the world to consider if the
SDGs are to be a success, and advances in digital technology, gender
equality, and climate action are major steps towards achieving those
goals.\footnote{``Population,'' United Nations (United Nations),
  accessed November 2, 2020,
  https://www.un.org/en/sections/issues-depth/population/index.html.} \\

\subsection{Bridging Africa's Digital Divide}

The World Bank has previously estimated that achieving the African
Union's goal for universal internet coverage, as is seen in most Western
societies, would increase GDP growth in Africa by 2 percentage points
annually. Additionally, the probability of becoming employed for
citizens of the country increases by 6.9 points when fast internet
becomes available.\footnote{Hafez Ghanem, ``Shooting for the Moon: An
  Agenda to Bridge Africa's Digital Divide,'' Brookings (Brookings,
  February 7, 2020),
  https://www.brookings.edu/blog/africa-in-focus/2020/02/07/shooting-for-the-moon-an-agenda-to-bridge-africas-digital-divide/.}
So, solving the African digital divide would go a long way also in
achieving a multitude of the SDGs as well as advancing society towards
the listed goals. \\

In 2019, there were 525 million internet users across Africa. While this
averages about 40\% of the entirety of the African continent, this usage
varies from country to country. Some countries see over 80\% of citizens
having access to the internet, but there are also a significant amount
that have internet access only for only the wealthy class of citizens or
government officials.\footnote{``Last Month, Over Half-a-Billion
  Africans Accessed the Internet,'' Council on Foreign Relations
  (Council on Foreign Relations), accessed November 2, 2020,
  https://www.cfr.org/blog/last-month-over-half-billion-africans-accessed-internet.} \\

\subsection{Reducing Gender Inequality}

The Global Partnership for Education calls gender inequality ``one of
the greatest threats to Africa's future.'' It also notes that without
significant action taken to mitigate the ongoing situation, the global
community will have ``failed a generation and a generation to come''
given that the SDGs would not be fully attained. This is especially true
given that there is a wide gap that still remains between genders on the
African continent, even as the rest of the world is as close to equality
as we have ever been. 70\% of women in Africa are completely excluded
financially, and there is currently a US\$42 billion gap between men and
women financially. Achieving a higher level of equality between women
and men could grow the continent's economy by approximately 10\%, adding
billions alone by the year 2025. \footnote{Victoria Egbetayo et al.,
  ``One of the Greatest Threats to Africa's Future: Gender Inequality,''
  Global Partnership for Education, December 16, 2019,
  https://www.globalpartnership.org/blog/one-greatest-threats-africas-future-gender-inequality.} \\

Further, the job and employment sectors are not the only areas in which
women are being excluded. Young girls living on the African continent
also face an uphill battle with millions who are not in school right
now, and it is entirely possible that millions will never have the
opportunity to set foot into a classroom. \footnote{ibid} If the SGDs
are to be achieved throughout the African continent and globally, it is
imperative that special attention is given to women's education and
employment opportunities. \\

\subsection{Climate Action: Promoting Responsible Growth}

UN Deputy Secretary-General Anima Mohammed describes climate change as a
``web factor that can lead to conflict.'' Climate change is one of the
great security challenges of the 21st century and can become a source of
future conflict as states struggle with access to water, food, and
energy. \footnote{Ash Murphy PhD Researcher. "Climate Change Is a
  Security Threat -- so Where Is the UN Security Council?" The
  Conversation. September 29, 2018. Accessed January 25, 2019.} The
Intergovernmental Panel on Climate Change recently reported that an
average global rise of 1.5 degree Celsius would put 20-30\% of species
at risk of extinction, while a 2 degree increase would cause most
ecosystems to significantly struggle.\footnote{Iisd. "UN Security
  Council Addresses Climate Change as a Security Risk \textbar{} News
  \textbar{} SDG Knowledge Hub \textbar{} IISD." SDG Knowledge Hub.
  Accessed January 25, 2019.
  http://sdg.iisd.org/news/un-security-council-addresses-climate-change-as-a-security-risk/.}
The Sahara is the world's largest desert, and it has the deepest layer
of intense heating compared to anywhere else on the globe. A 1.5 degree
increase in temperature would have disastrous effects for the nearly 2.5
million people living in the area. \\

In a continent where hundreds of millions of people depend on rainfall
for their food, it is apparent that a changing global climate will have
significant impacts on the millions of lives. The African continent is
home to many beautiful natural wonders, and they are at risk of being
damaged or destroyed forever if significant action is not taken to
reduce climate change.\footnote{`` How Africa Will Be Affected by
  Climate Change,'' BBC News (BBC, December 15, 2019),
  https://www.bbc.com/news/world-africa-50726701.} \\

On a global scale, the United Nations Security Council has taken
previous actions to address climate change occurring within the Lake
Chad basin region of the African continent. The UNSC adopted Resolution
2349 in March of 2017, and its main goal was to recognize that climate
change played a root cause in the conflict occurring in the region.
\footnote{Resolution 2349, accessed October 21, 2020,
  http://unscr.com/en/resolutions/doc/2349.} \\

Not only should we focus on taking efforts to preserve and protect the
African continent's beauty, we should simultaneously discuss what
efforts can be taken to grow the economy and bring people out of
poverty. Can both be achieved reasonably? \\

\subsection{Questions to Consider}
\begin{itemize}
\item 
How can Member Nations better preserve Africa's natural beauty while
simultaneously developing economically and industrializing? What stage
is your country in this growth process, and how has your nation
attempted to bridge that divide? 

\item 
Given Africa's rising population, how can we insure that all people
today as well as those yet to be born have access to even the most basic
of human necessities, and how can we insure that they are secured
sustainably? 

\item 
What actions and initiatives can be taken to successfully implement the
SDGs in your country and is it possible to attain them on time? How does
yours compare to other Member States? 

\item 
What role will the African Union play in a world where the highest
population growth is occurring on the African continent, and how can
sustainable development leave a better world for posterity? 

\item 
What would successfully implementing the sustainable development goals
and the African Union's Agenda 2063 look like for your nation, and what
should the African Union members have to look forward to for the future? 

\item 
What has your member state done to combat climate change and are those
practices transferable to Africa as a whole? 
\end{itemize}

\subsection{Vocabulary} 

\begin{itemize}
\item \textbf{Climate change mitigation} - efforts to reduce the effects of man-made
climate change on the environment

\item \textbf{Climate change adaptation} - ways by which nations are adapting and
evolving to face the challenges posed by climate change

\item \textbf{Sustainable} - A measure of the ability for something or some plan to be
maintained at a particular level, indefinitely

\item \textbf{Sustainable Development Goals} - 17 interlinked goals working as a
blueprint for advancing the socioeconomic status of UN Member States
that are LDCs

\item \textbf{LDCs} - Less Developed Countries.

\item \textbf{MDCs} - More Developed Countries.

\item \textbf{Millenium Development Goals} - Eight goals of development and improvement
decided upon by the United Nations with a target date of 2015. Three and
a half of these targets were achieved.

\item \textbf{National Development Plan} - Refers to domestic planning for
implementation of the SDGs

\item \textbf{Developing Country} - A state undergoing active, relatively rapid
economic development, may be lagging behind others.

\item \textbf{Poverty} - A dearth of material possessions to provide for basic needs.

\item \textbf{Sanitation} - Measure of cleanliness of public consumption resources

\item \textbf{Sustainability} - A measure of the ability for something or some plan to
be maintained at a particular level, indefinitely

\end{itemize}

\subsection{Key Organizations and Treaties}

\begin{itemize}
\item
\texttt{\href{https://unfccc.int/process/bodies/supreme-bodies/conference-of-the-parties-cop}{Conference of Parties Meeting (COP)}}

\item
\texttt{\href{https://www.undp.org/}{ The United Nations Development Programme}}
 
 \item
 \texttt{\href{https://www.ifpri.org/publication/womens-empowerment-agriculture-index}{Women's Empowerment in Agriculture Index (WEAI)}}
 
 \item
 \texttt{\href{https://www.un.org/sustainabledevelopment/progress-report/}{SDG Yearly Reports and Agendas}}
   
 \item
 \texttt{\href{https://sustainabledevelopment.un.org/index.php?page=view\&type=400\&nr=2051\&menu=35}{Addis Ababa Action Agenda}}
   
 \item
 \texttt{\href{https://www.unfpa.org/}{United Nations Population Fund}}
  
   \item
   \texttt{\href{https://population.un.org/wpp/}{United Nations Population Division}}
 \item
 \texttt{\href{https://www.data4sdgs.org/}{The Global Partnership for Sustainable Development Data}}
   
  \item
   \texttt{\href{https://www.ohchr.org/en/professionalinterest/pages/cedaw.aspx}{Convention on the Elimination of All Forms of Discrimination Against
  Women}}
 \end{itemize}

\newpage
\section{Topic B: Combatting Lasting Conflicts \& Securing
Lasting Peace for Africa}

\subsection{Introduction}

The African continent in recent years has been the scene of various
conflicts of almost every type and scale. It is no surprise, therefore,
that the African Union adopted the initiative called ``Silencing the
Guns by 2020'' in 2013 to help combat the negative effects conflict has
played across the African continent. The ultimate goal of the initiative
is to end all wars, conflicts, and genocides by the year
2020.\footnote{``Silencing the Guns by 2020,'' Silencing the Guns by
  2020 \textbar{} African Union, October 8, 2020,
  https://au.int/en/flagships/silencing-guns-2020.} \\

In February of 2019, the United Nations Security Council welcomed the
idea of the ``Silencing the Guns'' campaign. It acknowledged through
unanimous adoption of a resolution that only Africans can solve African
problems, and it highlighted the importance of other countries helping
to advance and accelerate progress. However, the Security Council also
recognized that efforts to create a continent free from conflict was a
daunting task, especially given the ``challenging security situation in
parts of Africa.''\footnote{``Silencing the Guns Campaign Kicks off in
  2020 \textbar{} Africa Renewal,'' United Nations (United Nations),
  accessed October 21, 2020,
  https://www.un.org/africarenewal/magazine/december-2019-march-2020/silencing-guns-campaign-kicks-2020.}
While this may be true, hope is not lost for this body to achieve this
goal in the near future. \\

\subsection{Combatting Terrorism}

For most of the world, particularly the Western world, the fight against
terrorism began on September 11, 2001 when the World Trade Centers in
New York City were attacked by Al-Qaeda militants. Terrorism in Africa,
however, did not start with those attacks. It began in Sudan in the
1990s, where Osama bin Laden organized an attack against Egyptian
president Hosni Mubarak. \footnote{John Harbeson, ``The War on Terrorism
  in Africa ,'' n.d.} \\

One of the primary instigators of terrorism on the African continent has
been the Boko Haram group, whose name translates to ``Western Education
is Forbidden.'' Boko Haram extremists currently reside in the northern
states of Nigeria, but they have been active across the continent in
encouraging violence.\footnote{``Boko Haram Fast Facts,'' CNN (Cable
  News Network, September 7, 2020),
  https://www.cnn.com/2014/06/09/world/boko-haram-fast-facts/index.html.}
More information on Boko Haram, as well as a timeline of attacks can be
found \texttt{\href{https://www.cnn.com/2014/06/09/world/boko-haram-fast-facts/index.html}{here}}.\\

Mohamed Ibn Chambas is the Head of the United Nations Office for West
Africa and the Sahel, and he recently said that despite intensive
efforts to prevent it, violent extremists have continued to attack
civilians and military operations in the region. He noted that
``terrorists continue to exploit latent ethnic animosities and the
absence of the State in peripheral areas to advance their agenda.'' He
also made mention that the ongoing COVID-19 crisis is strengthening the
root causes of conflict and aggravating already existing tensions
between groups, causing disproportionate impacts on the lives of women
and girls in conflict. \footnote{``Situation in West Africa, Sahel
  'Extremely Volatile' as Terrorists Exploit Ethnic Animosities, Special
  Representative Warns Security Council \textbar{} Meetings Coverage and
  Press Releases,'' United Nations (United Nations), accessed October
  21, 2020, https://www.un.org/press/en/2020/sc14245.doc.htm.} \\

\subsection{Combatting Current Conflicts}

While the ``Silencing the Guns'' initiative has taken broad steps to
help create a conflict-free Africa, it has also had several shortcomings
which need to be addressed in order to create a lasting peace. For
example, as the AU continues to create programs to tackle a wide variety
of issues, sourcing funds and manpower has become one of the most
critical challenges to overcome. Among other things, the initiative has
largely failed to stem the flow of arms and reduce the number of arms on
the continent. While it was a significant step towards creating a more
peaceful African continent, the success of the action depends solely on
people identifying the shortcomings and working to address
them.\footnote{``Silencing the Guns in Africa: Achievements and
  Stumbling Blocks,'' Africa Portal, accessed October 21, 2020,
  https://www.africaportal.org/features/silencing-guns-africa-achievements-and-stumbling-blocks/.}\\

Also, the body may choose to address more specific, ongoing points of
contention on the African continent. The body may focus on how to
address conflicts as a whole and include measures to implement in
specific regions such as the Democratic Republic of the Congo, Somalia,
South Sudan, Nigeria, and the Central African Republic. These are all
areas where tens of thousands of people have been killed and displaced
by ongoing violence. \footnote{``Work in Progress for Africa's Remaining
  Conflict Hotspots \textbar{} Africa Renewal,'' United Nations (United
  Nations), accessed October 31, 2020,
  https://www.un.org/africarenewal/magazine/december-2019-march-2020/work-progress-africa\%E2\%80\%99s-remaining-conflict-hotspots.}Therefore,
it would be in the best interest for delegates to focus on two parts of
this sub-topic: mitigating conflict as a whole in these regions as well
as assuaging the effects of the violence on the lives of the people
there. More information on these conflicts can be found \texttt{\href{https://www.un.org/africarenewal/magazine/december-2019-march-2020/work-progress-africa\%E2\%80\%99s-remaining-conflict-hotspots\#:~:text=In\%20its\%20quest\%20to\%20\%E2\%80\%9CSilence,Libya\%2C\%20where\%20tens\%20of\%20thousands}{here}}
 \\

\subsection{Measures to Prevent Conflicts and Secure Lasting Peace}

Not only should this body focus on methods to insure that current
conflicts are ended once and for all, it should also focus on how
conflict can be prevented for generations to come. One of the primary
purposes of the United Nations and its bodies is to maintain
international peace and security. For the African Union, this means
ensuring that African states are taking action to secure lasting peace
and cooperation with fellow African states for the benefit of posterity. \\

Building lasting peace on a continent that has recently been subject to
one violent conflict after another is a difficult challenge. Below you
will find resources on preventing conflicts and securing lasting peace
in order to draw up some ideas for how the body might address
peacebuilding as a whole. \\

\subsection{General Resources for Peacebuilding}
  \begin{itemize}
  \item \texttt{
    \href{https://www.un.org/africarenewal/magazine/august-2012/building-peace-ground}{Building peace from the ground up}}
  \item \texttt{
    \href{https://www.un.org/africarenewal/magazine/january-2013/after-africa\%E2\%80\%99s-wars-\%E2\%80\%98new-day\%E2\%80\%99-building-peace}{After Africa’s wars, a ‘new day’ for building peace}}
  \item \texttt{
    \href{https://www.accord.org.za/ajcr-issues/african-approaches-to-building-peace-and-social-solidarity/}{African Approaches to Building Peace and Social Solidarity}}
  \end{itemize}

\subsection{Questions to Consider}

\begin {itemize}

\item
What have been the successes and failures of the African Union's
``Silencing the Guns'' Initiative? How can we allow for its successes to
continue to succeed and in what way can we address and overcome its
shortcomings to achieve an African continent free from conflict?

\item
What are the key contributors to conflict and how has your nation
attempted to address them? Is there a particular program your country
has taken which has been successful, and could it be implemented across
the African Union?

\item
What steps can be taken to help mitigate current and ongoing conflicts
in Africa, and what is the African Union's role in these conflicts? What
are your nation's views on African Union intervention in regional
conflicts and disputes?

\item
How can the African Union ensure that when conflict is inevitable and
countries and regions devolve into violence that groups maintain the
rules of war and that citizens, particularly women and children,
continue to have access to basic necessities for life? How has your
nation attempted to safeguard citizens' rights in times of political and
economic turmoil?

\end{itemize}

\subsection{Vocabulary}

\begin{itemize}
\item 
\textbf{Blood Diamonds} - diamonds mind in an area of conflict and used to
finance violence and conflict, and is also referred to as conflict
diamond

\item
\textbf{Colonialism} - practice of obtaining control over a region, occupying it
with settlers, and exploiting it economically for the benefit of the
colonizer

\item
\textbf{Water Rights} - the right to use water in a particular area for a
particular purpose

\item
\textbf{Ceasefire} - temporary cessation of fighting, usually for purposes of
negotiating peace talks or a more permanent end to conflict

\item
\textbf{Armistice} - an agreement between two or more actors in a conflict to
stop fighting for a certain amount of time

\item
\textbf{War Crimes} - an action performed during the conduct of war that violates
the internationally accepted rules of war, generally accepted to be
defined according to the Geneva Convention on War

\item
\textbf{Crimes Against Humanity} - crimes specifically committed through a large
scale attack, usually targeting civilians

\item
\textbf{International Court of Justice} - the primary judicial organ of the
United Nations which settles disputes between states based on
international law

\item
\textbf{International Criminal Court} - an international judicial organ that
persecutes specific individuals charged with those considered to be the
most heinous crimes against the international community
\end{itemize}

\subsection{Key Organizations and Treaties}

\begin{itemize}
\item
  \texttt{
  \href{https://www.un.org/en/africa/osaa/pdf/au/cap_smallarms_2000.pdf}{African Common Position on the Illicit Proliferation, Circulation, and
  Trafficking of Small Arms \& Light Weapons}}
\item
 \texttt{
  \href{http://www.poa-iss.org/RegionalOrganizations/SADC/Instruments/SADC\%20Protocol.pdf}{Protocol on the Control of Firearms, Ammunition, and Other Related
  Materials}}
 \item
   
  \texttt{\href{https://www.un.org/disarmament/convarms/salw/}{Convention on Small Arms and Light Weapons}}
\end{itemize}

\newpage
\section{Topic C: Taking Measures to Reduce Corruption}

\subsection{Introduction}

It is a fairly well established fact that corruption across Africa
significantly hinders the economic, political, and social development of
the continent. Corruption is not just a government issue that causes
problems for government institutions. It affects real, everyday people
across nations, and it causes a significant detriment on their lives.
For some, the most basic services such as healthcare, police, and even
schools oftentimes require a bribe of a government official for a
citizen to get those most basic necessities. Overall, corruption is
placing a significant hindrance on the ability of the African Union to
bring more people out of poverty and to solve the wide ranging issues
affecting the continent.\footnote{``Citizens Speak out about Corruption
  in Africa - News,'' Transparency.org, accessed October 21, 2020,
  https://www.transparency.org/en/news/citizens-speak-out-about-corruption-in-africa.} \\

\subsection{Previous Action}

The African Union has adopted its own version of the United Nations'
Convention Against Corruption. The African Union Convention on
Preventing and Combating Corruption was adopted in 2003, and aims to
create ``mechanisms required to prevent, detect, punish, and eradicate
corruption and related offences in the public and private
sectors.''\footnote{``International Anti-Corruption Resources: Home,''
  GW Law Library: Library Guides, accessed November 2, 2020,
  https://law.gwu.libguides.com/c.php?g=187780.} The AUCPCC acts
effectively as a roadmap for other nations to adopt on the national
level, but the treaty has been ratified by only 42 of the 54 Member
States.\footnote{``AUABC in Brief,'' African Union Advisory Board on
  Corruption \textbar{} Mission and Vision of the Board, accessed
  November 2, 2020, http://www.auanticorruption.org/auac/en.} The
convention attempts to solve a number of corrupt offences including
bribery, diversion of property by public individuals, trade influence,
and money laundering.\footnote{``Egypt Joins African Union
  Anti-Corruption Treaty as per Presidential Decree,'' EgyptToday,
  August 23, 2020,
  https://www.egypttoday.com/Article/1/91148/Egypt-joins-African-Union-anti-corruption-treaty-as-per-presidential.} \\

Furthermore, the African Union chose 2018 as the year of winning in the
fight against corruption, making it their central theme and main idea
throughout. The entire purpose of setting that theme was to rally
African leaders behind this idea and encourage them to make more
progress in the fight against corruption. While many were skeptical of
the impact such a theme would have, there were a few actions done
throughout the year that could be classified as real progress. For
example, after the official launch of the theme in January of 2018,
African leaders met and acknowledged the severe negative impact that
corruption has on development across the continent, and they committed
to bringing about real action. Later that year, 37 African countries
signed an open letter to the African Union urging them to make strong
commitments and turn rhetoric into action. This was known as the
Nouakchott Declaration, and it served as an important policy stance on
the areas in need of the most attention, including establishing
country-by-country financial reporting, strengthening the African Tax
Administration Forum, and implementing transparent budget practices.
Finally, 2018 saw three more ratifications of the AUCPCC, and two others
expressed interest in formally ratifying it (Morocco and
Tunisia).\footnote{``The AU Needs to Walk the Talk on Corruption,'' The
  Mail \& Guardian, February 9, 2019,
  https://mg.co.za/article/2019-02-09-00-the-au-needs-to-walk-the-talk-on-corruption/.} \\

However, despite these actions, the African Union itself has previously
been accused of the same corruption that they seek to prevent. A leaked
internal memo from 2020 and written by AU staff described the
organization as run like a ``mafia-style cartel'' and controlled by
corruption and cronyism. The memo also lists several examples of
allegedly illegal staff appointments, and it demands the ``restoration
of sound administrative management and leadership.''\footnote{``African
  Union Strongly Denies Allegations of Cronyism, Corruption,'' Voice of
  America, accessed November 2, 2020,
  https://www.voanews.com/africa/african-union-strongly-denies-allegations-cronyism-corruption.}
This was certainly not the first time the African Union was accused of
corruption, however. In 2018, a member of the African Union's
Anti-Corruption Advisory Board, established by the AUCPCC, resigned in
fury after three years of witnessing ``multiple irregularities'' within
the organization that are significantly hindering the ability to fight
corruption. He said in his resignation letter that abuse of power and
lack of transparency and accountability have significantly harmed his
work.\footnote{``African Union Strongly Denies Allegations of Cronyism,
  Corruption,'' Voice of America, accessed November 2, 2020,
  https://www.voanews.com/africa/african-union-strongly-denies-allegations-cronyism-corruption.} \\

These accusations of high level corruption within organizations charged
with managing and controlling such practices have led the media to ask a
rather important question: is the African Union doing enough to tackle
corruption?\footnote{TRTWorld, Is the African Union doing enough to
  tackle corruption and violence? (TRT World, November 16, 2018),
  https://www.trtworld.com/africa/is-the-african-union-doing-enough-to-tackle-corruption-and-violence-21693.} \\

\subsection{Current Situation}

A poll conducted by Transparency International found that more than half
of all citizens in African countries think corruption is getting worse
within their government and that the government is doing a poor job in
managing corruption. The group found that one in every four people who
accessed public services in the last year had paid a bribe. If this data
is representative of the true percentage of African citizens who paid a
bribe for basic government services, that equates to approximately 130
million citizens in 35 countries.\footnote{``Citizens Speak out about
  Corruption in Africa - News,'' Transparency.org, accessed October 21,
  2020,
  https://www.transparency.org/en/news/citizens-speak-out-about-corruption-in-africa.} \\

Moreover, corruption does not affect all people equally across different
demographics and backgrounds. Corruption impacts the poor more than the
wealthy in Africa. According to Transparency International's report, the
poor are more than twice as likely to pay a bribe for a basic service.
The fact that the poor are adversely affected by corruption leaves
significantly less money for families to pay for other basic necessities
such as food, water, and medication. Further, the young African
population aged 18-34 years old are much more likely to pay bribes than
those aged over 55 years of age based on that report. This means that
those who are looking to start a family and raise children are much more
likely to be denied access to education and healthcare, which can have
life altering effects.\footnote{ibid.} \\

National governments are not the only worry in terms of corrupt deeds
taking place across the continent. There has been increasing corruption
occurring with international businesses and corporations. Non-African
actors play a large part in corruption through foreign bribery and money
laundering. International companies participate in bribery to receive
better deals with officials and governments or to gain the best
government contracts. \footnote{``Where Are Africa's Billions? - News,''
  Transparency.org, accessed October 21, 2020,
  https://www.transparency.org/en/news/where-are-africas-billions.} \\

However, this situation does not lack hope for the future. 53\% of those
polled by Transparency International agreed that they were hopeful for
the future and that ordinary, everyday people can lead the fight against
corruption even in areas where the government refuses to take action.
\footnote{``Citizens Speak out about Corruption in Africa - News,''
  Transparency.org, accessed October 21, 2020,
  https://www.transparency.org/en/news/citizens-speak-out-about-corruption-in-africa.} \\

\subsection{Questions to Consider}

\begin{itemize}
\item
How can the African Union better provide an outlet by which concerned
citizens can express and report their concerns and experiences with
bribery and other corrupt acts, without fear of repercussion? What
should the African Union's response to reported corruption be? 
\item
What role does the media and civil society hold in putting pressure on
governments to be more transparent and draw attention to corruption? How
has your nation protected the freedom of the press, and what are some
examples of times the press uncovered corrupt acts by government
officials? 
\item
How can the African Union assure the independence and freedom of the
media and press and what role do they play in highlighting corruption
and wrongdoing in government? 
\item
How can the African Union better prevent foreign non-African actors from
intervening in the affairs of African states and encouraging corrupt
acts? 
\item
Does your nation presently have anti-corruption systems in place? What
has been their success and failures, and could similar measures be
implemented across the African Union? 
\item
How should the African Union weigh the benefit of nongovernmental
organizations as well as foreign investment alongside the detriment they
have in encouraging institutions of bribery and political misconduct?
How has your country been influenced by this? 
\item
Has your country adopted or signed the African Union Convention on
Combating and Preventing Corruption? If not, why? If so, how far along
have you come in fully implementing the guidelines built by the
agreement? 
\end{itemize}

\subsection{Vocabulary}

\begin{itemize}
\item 
\textbf{Non-governmental organization (NGO)} - a non-profit group who functions
outside the scope of any entity or body related to governments, usually
advocating for a humanitarian or environmental cause
\item
\textbf{Corruption Perceptions Index (CPI)} - an index created by Transparency
International which ranks 180 countries on their perceived level of
corruption, and scores them between 0 and 100, with 0 being highly
corrupt and 100 being highly clean
\item
\textbf{Political Corruption} - Use of powers granted to government officials in
an illegal manner or for illegitimate personal gain.
\item
\textbf{Bribery} - Giving or receiving a material or monetary gift, illegally, in
exchange for exercise of vested authority
\item
\textbf{Extortion} - Acquiring material goods or power through force or threat of
force
\item
\textbf{Cronyism} - Appointing friends or acquaintances to positions of power, or
providing power, regardless of qualifications
\item
\textbf{Nepotism} - Favoring relatives or close friends in the exercise of
authority and power, including through material gifts, political
appointments, jobs, etc.
\item
\textbf{Parochialism} - Narrow outlook and limited focus; not seeing the big
picture
\item
\textbf{Patronage} - Control of political appointments or given rights to
political privileges
\item
\textbf{Influence Peddling} - Employing a position of political authority or
power in exchange for money, material goods, or favors
\item
\textbf{Graft} - Use of political authority for personal gain, typically through
misdirection of public funds in order to enrich or otherwise benefit the
individual or private interests
\item
\textbf{Embezzlement} - Usage of funds for purposes other than intended,
dishonestly
\item
\textbf{Dictatorship} - Autocratic government under a single authority figure
\item
\textbf{Oligarchy} - A small collection of individuals having control over a
government
\item
\textbf{Junta} - Military or political entity that controls a country's
government after seizing power through force
\item
\textbf{Coup or Coup d'etat} - Illegal seizure of power from a government
\end{itemize}

\subsection{Treaties and Organizations}

\begin{itemize}
\item
\texttt{ 
  \href{https://www.unodc.org/unodc/en/treaties/CAC/}{The United Nations Convention Against Corruption}}
  \item
  \texttt{ 
  \href{http://www.oecd.org/corruption/oecdantibriberyconvention.htm}{OECD Convention Against Foreign Bribery}}
  \item
  \texttt{ 
  \href{https://www.coe.int/en/web/greco}{Group of States Against Corruption}}
  \item
  \texttt{ 
  \href{https://www.unodc.org/}{The United Nations Office on Drugs and Crimes}}
  \end{itemize}

\end{document}
