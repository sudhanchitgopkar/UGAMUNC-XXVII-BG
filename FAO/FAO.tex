\documentclass[10pt, letterpaper]{article}
        \usepackage[utf8]{inputenc}
        \usepackage[margin=1in]{geometry}
        \usepackage{fancyhdr}
        \usepackage{titling}
        \usepackage{enumitem}
        \usepackage{mathtools}
        \usepackage{amssymb}
        \usepackage{xfrac}
        \usepackage{booktabs}
        \usepackage{graphicx}
        \usepackage{wrapfig, blindtext}
        \usepackage{hyperref}
        \usepackage{enumerate}
        \usepackage{multicol}

        
        \setlength{\parindent}{0pt}

\title{Journal 2}
        \author{Sudhan Chitgopkar}
        \date{\today}
        
        % headers -- no need to change
        \pagestyle{fancy}
        \fancyhf{}
        \lhead{FAO}
        \chead{UGAMUNC XXVII}
        \rhead{\thedate}


\begin{document}

Dear Delegates, \\

It is my absolute pleasure to welcome you all to the UGAMUNC XXVII Food
and Agriculture Organization (FAO) committee! My name is Reid Cogswell,
and I will have the pleasure of serving as your chair at this year's
conference. I am from Alpharetta, Georgia, and I am a junior here at the
University of Georgia majoring in Management Information Systems with a
minor in Computer Science. Model United Nations has become a huge part
of my life since coming to college and has really helped me come out of
my shell and shaped me into the person I am today! In addition to MUN, I
am an active participant in UGA's UGArden club. When I have free time I
enjoy watching new shows, hanging out with friends, and learning about
new things. \\

I also have the immense pleasure of getting to introduce you all to my
co-chair, George Moore. George is from Toccoa, Georgia, and he is a
freshman studying Economics and International Affairs with a minor in
Spanish and certificate in International Agriculture. This is George's
first year participating in Model UN. He serves as a First-Year Senator
with SGA and as President of Russell Hall Community Council. He also
enjoys being a part of the Baptist Collegiate Ministry here at UGA and
attending Prince Avenue Baptist Church in Athens. George's favorite
things include politics, talk radio, Diet Coke, traveling, expanding his
comfort zone, and meeting new people. George also enjoys reading books
about Christianity, theology, and United States history. He is excited
to be co-chairing the Food and Agriculture Organization committee at his
first UGAMUNC this year! \\

As a delegate of our committee, it is expected that you compete to the
best of your ability and prepare adequately. Although we are excited to
see hard fought debate among you all on the weekend of the conference,
we would like to make it clear that we expect the highest level of
diplomacy and professionalism between delegates at all times. We invite
you to use this background guide as the foundation of your research of
the topics that shall be presented to you for the committee, and to use
it as a stepping stone in order to analyze and research each topic to
its fullest extent. When writing your position paper, we ask that you
focus on the scope in which your member states are affected by these
topics and that you are accurate with the strategies that you present
while representing them in the FAO committee. Ask yourself questions
while writing such as, what they are doing now in relation to the topic,
and what are they/could they do in the future to address these issues?
This remains true with your contributions through speeches and
resolution writing as well, but to reiterate, your contributions should
align with your member state's views and leanings to the issues at hand.
We expect final resolutions to reflect the cooperation of member states
and that they align with the goals and aspirations of the FAO committee
as a whole. \\

Should you have any questions prior to the committee, feel free to reach
out to me about anything! My email is \texttt{\href{mailto:arc73290@uga.edu}{arc73290@uga.edu}}. Please submit
your completed position papers to my co-chair, George
\texttt{\href{mailto:ghm23035@uga.edu}{(ghm23035@uga.edu)}}, and I \textbf{by February 1st 11:59 PM}. We urge you to come
to our first session well rested and prepared for a great weekend. \\

Reid Cogswell \& George Moore

\newpage
\tableofcontents
\newpage

\section{Background}

The Food and Agriculture Organization (FAO) is a specialized agency in
the United Nations whose goals have been the elimination of hunger, the
improvement of nutrition, and the improvement of the standard of living
by increasing agricultural productivity.\footnote{Karen Mingst, ``Food
  and Agriculture Organization,'' Encyclopædia Britannica (Encyclopædia
  Britannica, inc., July 31, 2006),
  https://www.britannica.com/topic/Food-and-Agriculture-Organization.}
The FAO has held these objectives since October 1945 when the agency was
founded and headquartered in Washington, D.C, but was later moved to
Rome, Italy in 1951.\footnote{Ibid.} Throughout its history, the FAO has
coordinated its efforts with that of governments and technical agencies
around the world in the development of programs in agriculture,
forestry, fisheries, and land and water resources. According to
Britannica, the FAO ``carries out research; provides technical
assistance on projects in individual countries; operates educational
programs through seminars and training centres; maintains information
and support services, including keeping statistics on world production,
trade, and consumption of agricultural commodities; and publishes a
number of periodicals, yearbooks, and research bulletins''.\footnote{Ibid.} \\

Throughout the FAO's history as the oldest specialized agency in the
United Nations, it has contributed to much of the progress and
accomplishments made in agricultural development and food security. One
example of such achievements is the codex alimentarius, which was
created by the FAO to set international food standards and help ensure
food safety in international trade. Another is the eradication of
Rinderpest, which was a viral disease that killed millions of livestock
and caused famine among global citizens.\footnote{\emph{10 Achievements
  of the Food and Agriculture Organization of the United Nations},
  \emph{Youtube}, 2017,
  https://www.youtube.com/watch?v=wYxMwaTB\_AQ\&ab\_channel=FoodandAgricultureOrganizationoftheUnitedNations.}
These are just a few of the multiple notable achievements that the FAO
has made in its long history under the United Nations. \\

Today, the FAO has over 194 member states within its body and operates
in 130 countries across the globe in an effort to achieve their
goals.\footnote{``About FAO,'' Food and Agriculture Organization of the
  United Nations, 2020, http://www.fao.org/about/en/.} Currently, the
FAO uses its five Strategic Objectives, which are laid out in their 2030
Agenda for Sustainable Development, to address the ever increasing
demands of agricultural development.\footnote{``What We Do : FAO: Food
  and Agriculture Organization of the United Nations,'' FAO, 2020,
  http://www.fao.org/about/what-we-do/en/.} These five Strategic
Objectives include helping in the elimination of hunger, food
insecurity, and malnutrition; making agriculture, forestry, and
fisheries more productive and sustainable; reducing rural poverty;
enabling inclusive and efficient agricultural and food systems; and
increasing the resilience of livelihoods to threats and
crises.\footnote{Ibid.} \\

\newpage
\section{Topic A. Exploring the use of Biotechnology to Eliminate Food
Insecurity}

\subsection{Introduction}

Food insecurity is a battle that the Food and Agriculture Organization
(FAO) has been fighting since its creation back in 1945. According to
the United Nations, ``it is estimated that over 2 billion people do not
have regular access to safe, nutritious and sufficient food, including 8
percent of the population in North America and Europe,''\footnote{``Food,''
  United Nations (United Nations, 2019),
  https://www.un.org/en/sections/issues-depth/food/index.html.} not to
mention that in 2018, the UN estimated that around 821 million people
across the world were undernourished.\footnote{Ibid.} With such a large
portion of the human population struggling with food insecurity,
scientists have turned to biotechnology as a way to create a sustainable
way to produce for these 2 billion people affected by food insecurity by
2050.\footnote{``Biotechnology,'' Food and Agriculture Organization of
  the United Nations, 2020, http://www.fao.org/biotechnology/en/.} \\

Biotechnology, when referring to plant agriculture, is used to improve
crop insect resistance, enhance crop herbicide tolerance, and to help
facilitate the use of more environmentally sustainable farming
practices, whereas biotechnology in animal agriculture is referred to as
genetically engineering animals to improve their suitability for
pharmaceutical, agricultural, or industrial applications.\footnote{``Food
  \& Agricultural Biotechnology,'' BIO, accessed October 31, 2020,
  https://archive.bio.org/food-agricultural-biotechnology.} That being
said, there is debate among world consumers as to the risk and safety in
using biotechnology to genetically modify plants and livestock.
Therefore, the FAO has been working hand in hand with international
organizations and Member States alike when outlining how to implement
science-based safety evaluations and risk assessment systems, giving
recommendations on how to properly label and distribute genetically
modified organisms (GMOs).\footnote{``FAO.org,'' Biotechnology
  \textbar{} Food safety and quality \textbar{} Food and Agriculture
  Organization of the United Nations, 2020,
  http://www.fao.org/food-safety/scientific-advice/crosscutting-and-emerging-issues/biotechnology/en/.} \\

\subsection{History}

The use of biotechnologies is not a new concept for mankind, as we can
see the first glimpses of the application and discovery of the
crossbreeding of livestock and plants hundreds of years into our
past.\footnote{Ashish Swarup Verma et al., ``Biotechnology in the Realm
  of History,'' Journal of pharmacy \& bioallied sciences (Medknow
  Publications Pvt Ltd, July 2011),
  https://www.ncbi.nlm.nih.gov/pmc/articles/PMC3178936/.} However it was
not until the 1900s, with multiple developments in the scientific world,
that the international community began to see the true potential of
biotechnologies. Since the early 1980s, biotechnologies have been put
into practice specifically in livestock in order to accomplish increased
growth rates, enhanced lean muscle mass, enhanced resistance to disease
or improved use of dietary phosphorus to lessen the environmental
impacts of animal manure, etc.\footnote{``Information about Topics and
  Careers in Bioscience for Teachers, Students and Everyone Else,''
  About Bioscience, November 27, 2017,
  https://www.aboutbioscience.org/topics/animal-biotechnology/.} It is
worth noting though that, as of right now, there is only one genetically
modified livestock up for commercial sale and consumption: the
AquAdvantage salmon.\footnote{Center for Veterinary Medicine,
  ``AquAdvantage Salmon,'' U.S. Food and Drug Administration (FDA, April
  15, 2020),
  https://www.fda.gov/animal-veterinary/animals-intentional-genomic-alterations/aquadvantage-salmon.} \\

As far as genetically modified crops are concerned, in 1994, the world
saw the first genetically modified crop being sold commercially to
consumers for consumption.\footnote{Clive James and Anatole F.
  Krattiger, ``Global Review of the Field Testing and Commercialization
  of Transgenic Plants: 1986 to 1995,'' ISAAA (The International Service
  for the Acquisition of Agri-biotech Applications (ISAAA), 1996),
  http://www.isaaa.org/kc/Publications/pdfs/isaaabriefs/Briefs\%201.pdf.}
Following the first GM crop hitting the market, there were many more
significant applications of genetically modified foods being approved
for commercial use and sale across the globe. By 2010, 60 different
countries across the world had granted approvals to import crops made
from biotechnologies, with 29 of those countries growing commercialized
biotech crops. \footnote{``ISAAA Brief 43-2011: Executive Summary,''
  Executive Summary: Global Status of Commercialized Biotech/GM Crops:
  2011 - ISAAA Brief 43-2011 \textbar{} ISAAA.org, 2011,
  http://www.isaaa.org/resources/publications/briefs/43/executivesummary/default.asp.} \\

\subsection{Current Situation}

In current times, the FAO recognizes the need of biotechnology across
the world; itstates: ``when appropriately integrated with other
technologies for the production of food, agricultural products and
services, biotechnology can be of significant assistance in meeting the
needs of an expanding and increasingly urbanized
population''.\footnote{``Biotechnology,'' Food and Agriculture
  Organization of the United Nations, 2020,
  http://www.fao.org/biotechnology/en/.} In order to assist Member
States of the body, the FAO says that it has been ``providing them with
legal and technical advice, assisting them to develop their capacities
in agricultural biotechnologies and related issues through technical
co-operation and training, and providing them with access to
high-quality, updated, balanced, science-based information''.\footnote{Ibid.} \\

Currently, 191.7 million hectares of biotech crops have been planted by
up to 17 million farmers in 26 countries in 2018 alone. This is a
staggering change from the initial planting of 1.7 million hectares when
the first biotech crop was commercialized in 1996.\footnote{``Pocket K
  No. 16: Biotech Crop Highlights in 2018,'' Biotech Crop Highlights in
  2018 ISAAA.org, 2018,
  http://www.isaaa.org/resources/publications/pocketk/16/.} Regardless
of the growth in the industry throughout the years, the UN claims that
an estimated 821 million people suffered from hunger in 2018, and that,
without significant change, the Zero Hunger Target will not be achieved
by 2030. Meanwhile, the number of overweight and obese people continues
to increase worldwide..\footnote{Ibid.} That being said, some feel
biotechnology can only truly be effective in agriculture if small
farmers in all member states of the world can gain access to the
benefits of biotechnology, due to the tendency of uneven adoption of
technologies across countries and across sectors, leaving out those who
need them most.\footnote{``Realizing the Potential of Agricultural
  Biotechnology in the Asia-Pacific Region,'' Food and Agriculture
  Organization of the United Nations (FAO, 2019),
  http://www.fao.org/3/ca5106en/ca5106en.pdf.} \\

Some glaring issues that come to mind when discussing the implementation
of biotechnology across the world, according to the FAO, are that ``many
of the smaller countries in the region do not have the resources to
pursue such approaches on their own, and existing regional mechanisms
for collaboration and technology transfer will need to be strengthened.
International organizations like FAO and the International Fund for
Agricultural Development will be important partners in these
efforts''.\footnote{Ibid.} These are important to keep in mind as many
Low-to-Middle Income Countries (LMIC) will not be able to afford these
technologies on their own, and they are the ones who need it the most as
well. \\

\subsection{Conclusion}

Around the world, biotechnology is a very new concept which has been
developing for the last 24 years at a staggering rate. In theory, it is
the only way that scientists believe we will be able to hit the UN's
goal of Zero Hunger Target by 2030. However, there have been many
concerns about how effective a method biotechnologies can be, due to
LMIC's being unable to afford to implement it into their budgets. This
is why the committee must explore different ways in order to effectively
approach how the FAO can help these Member States, as well as any
outside help through various different international, private, and
public institutions. \\

\subsection{Committee Directive}

The goal of the Food and Agriculture Organization for this topic is to
explore how biotechnology is being used today to determine whether or
not it will be useful in the eradication of food insecurity. It is
expected that you are to formulate solutions that are sustainable in the
long term in the pursuit of this objective, and discuss which is the
best approach for all Member States of the body. While some Member
States of the body may not be affected by food insecurity as much as
others, it is important that the committee properly address solutions
for the Member States that are achievable. Use your knowledge of
biotechnology in the food and agriculture sector to think of creative
ways to address the situation. Keep in mind that biotechnology and the
FAO oversee crops, livestock, forestry, fisheries and aquaculture and
agro-industries so do not limit your solutions to just one of these
topics. \\

\subsection{Questions to Consider}

\begin{itemize}
\item
  
  How should the FAO approach the implementation of biotechnology in
  Member States who have not explored the use of biotech?
  
\item
  
  Are there new regulations that we should implement in order to ensure
  the safety in the use of products of biotechnology?
  
\item
  
  How does food insecurity affect the country you are representing? How
  does it affect others?
  
\item
  
  How can we further the research into biotechnology in order to achieve
  our goals?
  
\item
  
  How can biotechnology help the country you are representing?
  
\end{itemize}

\newpage
\section{Topic B. Promoting Agro-Industries in the Low and Middle-Income
Countries (LMIC) to Help Eradicate Poverty}

\subsection{History}

Agro-industry is an idea that has not always been widely accepted, and
the positive and negative effects are evident throughout history. ``The
double revolution in agriculture and industry, which occurred in England
in the 18th century, laid the foundation for the development of the
agro-food industry, and completely changed the conditions for
agriculture and food-production.''\footnote{http://www.museum.agropolis.fr/english/pages/expos/fresque/module\_15.htm}
The concept of converting raw materials into commodities has existed for
thousands of years, but as technology advances, agro-industry also
increasingly changes in both production and methods. Agro-industry
provides an escape from the economically crippling cycle that many LMIC
remain stuck in. When LMIC do not have to import the commodities that
they have the raw resources to produce, they can escape this cycle of
economic strife. Poverty plagues these nations, and if they could export
processed commodities then they could gain economic success.
Agro-industries have been used throughout history to help countries
transform their raw materials into processed commodities that they can
export to other countries for a profit. Agro-industry has been the way
that many countries have bounced back and gained control of their
economy. \\

Historically, agro-industry has lifted nations out of the grip of
poverty and allowed countries to become more self-reliant. Developed
countries have historically taken control over agro industry because
they have the financial resources that can change these raw materials to
commodities. Over history, agro-industry has been used to transform raw
materials into something that can be used, but as technology advanced
agro-industry began to change rapidly. Agro-industries have been around
for years and this idea of an agro-industrial complex took off in
Bulgaria in the 1970s. The agro-industries movement was started before
the second World War, and, in Bulgaria specifically, it was able to help
boost the country's power and financial resources, which often go hand
in hand. ``A
\texttt{\href{https://www.britannica.com/topic/cooperative}{cooperative}}
movement in agriculture developed before World War II. After the war,
cooperative farms were established in the fashion of Soviet
\texttt{\href{https://www.britannica.com/topic/kolkhoz}{kolkhoz}} on most arable
land. The cooperative and state farms later merged into large state and
\texttt{\href{https://www.britannica.com/topic/collectivization}{collective
units}}. These were further consolidated in 1970--71 into even larger
groupings, called agro-industrial complexes, that took advantage of
\texttt{\href{https://www.merriam-webster.com/dictionary/integrated}{integrated}}
systems of automation, supply, and marketing.''\footnote{\texttt{\href{https://www.britannica.com/place/Bulgaria/Economy\#ref476393}{\underline{https://www.britannica.com/place/Bulgaria/Economy\#ref476393}}}}
Agro-industry is changing constantly, and agro-industry looks completely
different today than it has before. While agro-industry changes rapidly,
the concept of transforming these raw materials into something that
people can use and export remains the same.

\subsection{Current Situation}

Developments in agro-industry have the potential to eradicate poverty if
done sustainably. This is the main component that is lacking with
traditional agro-industry. Current efforts include the African Union
Commission and the Zero Hunger vision by 2030. Agricultural
mechanization in Africa is an integral part of the Zero Hunger
vision.\footnote{http://www.fao.org/3/CA1136EN/ca1136en.pdf} There is a
historical struggle at making the agro-industry sustainable. Current
initiatives involve combating this sustainability struggle. \\

Also, agro-industry poses problems with equity, sustainability, and
inclusivity. Agro-industries must be sustainable, and in order for this
to occur they must be competitive in terms of costs, prices, operational
efficiencies, product offers, and other associated
parameters.\footnote{http://www.fao.org/3/i3125e/i3125e00.pdf} The
overall Bank Group Vision for Agriculture and Rural Development's
central goal is 
\texttt{\href{https://www.afdb.org/en/topics-and-sectors/topics/poverty-reduction/}{poverty
reduction}} and, agriculture and rural development are prime building
blocks.\footnote{https://www.afdb.org/en/topics-and-sectors/sectors/agriculture-agro-industries}
For agro-industry to be as effective as possible, efforts need to be
made to ensure that equity and sustainability are a priority.
Agriculture is the means of survival for many people in extremely poor
nations. Agro-industry can eradicate poverty if implemented effectively.
The World Development Report 2008 (World Bank, 2007) called attention to
the fact that some 800 million people are considered poor, subsisting
with incomes of less than the US \$1 per day. Among the world's poor,
75\% live in rural areas, having agriculture as a major source of
livelihood. Fighting poverty will require that economic growth and
development are brought to rural areas. Agro-industries are part of the
answer to this challenge.''\footnote{Ibid} With the implementation of
sustainable agro-industries Low and Middle-Income Countries have
numerous raw agricultural materials, and they survive off agriculture.
Promoting agro-industries at an unprecedented level can help eradicate
poverty if done with regards to the small farmer and sustainability. \\

\subsection{Conclusion}

The Food and Agricultural Organization is meeting in order to create
methods that can produce sustainable agroindustry with regards to the
small farmer. It is the committee's job to plan the implementation of
this to ensure that change takes place. The establishment and promotion
of agro-industry in LMIC will eradicate poverty by creating jobs and
giving LMIC the ability to take control over their resources and what
they import and export. Without agro-industry, nations are stuck in the
cycle of having to export their raw resources and import these same
resources after they have been processed and converted to commodities.
These LMIC become trapped in this cycle, which hinders economic growth
when they do not create an agro-industry. \\

\subsection{Committee Directive}

The Food and Agriculture Organization of the United Nations states that
``Agro-industries provide a means of converting raw agricultural
materials into value-added products while generating income and
employment and contributing to overall economic development in both
developed and developing countries.''\footnote{http://www.fao.org/biotech/sectoral-overviews/agro-industry/en/}
While agro-industries have the potential to turn economies of nations
around, there are many elements of agroindustry that can be harmful to
individual farmers. Agroindustry is consistently not sustainable, and
there are numerous struggles that small farmers face. ``Establishing and
maintaining competitiveness constitute a particular challenge for small-
and medium-scale agro-industrial enterprises and smaller-scale farmers.
Although agro-industries have the potential to provide a reliable and
stable outlet for farm products, the need to ensure competitiveness
favours farmers who are better able to deliver larger quantities and
better quality of products. To the extent that smaller, resource-poor
farmers are left out of supply chains, the socio-economic benefits of
agro-industries are potentially reduced. A need thus exists for policies
and strategies that, while promoting agro-industries, take into account
issues of competitiveness, equity and inclusiveness.''\footnote{http://www.fao.org/fileadmin/user\_upload/ags/publications/EEA\_light.pdf}
If done right the promotion of agroindustry can create jobs, grow the
economy, and bolster industry in LMIC. It will be your job to do just
that and create the way to promote sustainability in order to eradicate
poverty. \\

Delegates, given these issues you are tasked with creating ways to
combat these negative effects and creating a sustainable means of
agroindustry that can effectively eradicate poverty in LMIC. You are
also tasked with the implementation of these sustainability methods to
ensure that real change is creating from this assembly. With the shift
to agro-industries then farmers will be able to move from subsistence
farming to market-oriented farming while creating a business and greater
profit out of their agricultural yield. Sustainable agricultural
mechanization is key to ensuring that agro-industries are effective and
helping the maximum number of people possible. Over the course of this
committee you will work together to eradicate poverty through the
creation and implementation of agro-industries in LMIC. \\

\newpage
\section{Topic C. Facilitating the mitigation of Climate Change in the
Agricultural Sector}

\subsection{Introduction}

The agricultural sector of the world has been a major contributor to
climate change for years. The destructive nature of climate change has
affected farming practices around the globe with rising temperatures,
increased temperature variability, changes in levels and frequency of
precipitation, a greater frequency of dry spells and droughts, the
increasing intensity of extreme weather events, rising sea levels, and
the salinization of arable land and freshwater.\footnote{``THE STATE OF
  FOOD AND AGRICULTURE: CLIMATE CHANGE, AGRICULTURE AND FOOD SECURITY,''
  FAO ( Food and Agriculture Organization of the United Nations, 2016),
  http://www.fao.org/3/a-i6030e.pdf.} Climate change's effects on
agriculture will only continue to worsen if left ignored by the Food and
Agriculture Organization (FAO). The effects of climate change will
vastly reduce the amount of of crops we will yield annually, create
harsher conditions for our livestock to live, affect the productivity of
fisheries that produce nearly 50 percent of animal proteins for
Low-to-Middle Income Countries (LMIC) across the globe, and put our
forests at risk.\footnote{Ibid.} \\

While climate change is a troubling issue for the agriculture sector
across the globe, it is also a major contributor. In fact, agriculture
is the second biggest contributor to climate change, globally making up
roughly 21 percent of greenhouse gas emissions.\footnote{Ibid.} Despite
this, the issue of mitigating the effects of climate change is not
easily answered. This is because with the introduction of new policy, it
could potentially jeopardize food security and poverty reduction in
nations across the world, so it is important to keep the social and
economic impact on LMIC's in mind while creating solutions. \\

\subsection{History}

Throughout the times from the pre-industrial era to now, the world has
shown evidence of the disastrous effects of climate change. Since the
1750's we have seen a tremendous increase in the global atmospheric
concentrations of carbon dioxide, methane, and nitrous oxide. In the
time period between the late 1800's to the early 2000's we have seen an
increase in average temperature of 0.76 degrees.\footnote{Ibid.}
According to the World Meteorological Organization, ``at continental,
regional, and ocean basin scales, numerous long-term changes in climate
have been observed. These include changes in Arctic temperatures and
ice, widespread changes in precipitation amounts, ocean salinity, wind
patterns and aspects of extreme weather including droughts, heavy
precipitation, heat waves and the intensity of tropical cyclones. More
intense and longer droughts have been observed over wider areas since
the 1970s, particularly in the tropics and subtropics''.\footnote{``Climate
  Change and Desertification,'' World Meteorological Organization (World
  Meteorological Organization), accessed October 31, 2020,
  https://library.wmo.int/doc\_num.php?explnum\_id=5047.} We were able
to more accurately identify this in the 1990s when scientific research
on climate change had expanded and research had expanded our
understanding between links with historic data and the ability to model
climate change numerically. Research during this was directed and
summarized by the United Nations Intergovernmental Panel on Climate
Change (IPCC).\footnote{``History of the IPCC,'' IPCC, 2020,
  https://www.ipcc.ch/about/history/.} \\

Another UN body, the United Nations Framework Convention on Climate
Change UNFCCC), historically stated, ``responses to climate change
should be coordinated with social and economic development in an
integrated manner with a view to avoiding adverse impacts on the latter,
taking into full account the legitimate priority needs of developing
countries for the achievement of sustained economic growth and the
eradication of poverty,'\footnote{``THE STATE OF FOOD AND AGRICULTURE:
  CLIMATE CHANGE, AGRICULTURE AND FOOD SECURITY,'' FAO ( Food and
  Agriculture Organization of the United Nations, 2016),
  http://www.fao.org/3/a-i6030e.pdf.} which remains true, and has been a
talking point of the agricultural sector still to this day. \\

\subsection{Current Situation}

Current moves by the FAO in order to tackle this issue is a policy they
created called Climate-Smart Agriculture (CSA). According to the FAO,
``the CSA approach has three objectives: sustainably increasing
agricultural productivity to support equitable increases in incomes,
food security and development; increasing adaptive capacity and
resilience to shocks at multiple levels, from farm to national; and
reducing greenhouse gas emissions and increasing carbon sequestration
where possible''.\footnote{Ibid.} The introduction of CSA presents
multiple ideas on how to become more climate smart in farming practices,
such as the improvement of water harvesting and retention and water-use
efficiency. These are fundamental for the future of agriculture as
increasing production when addressing increasing irregularity of
rainfall patterns is key. These irrigation practices are important as
the CSA states, ``Today, irrigation is practiced on 20 percent of the
agricultural land in developing countries but can generate 130 percent
more yields than rain-fed systems. The expansion of efficient management
technologies and methods, especially those relevant to smallholders is
fundamental''.\footnote{``\,`Climate-Smart' Agriculture Policies,
  Practices and Financing for Food Security, Adaptation and
  Mitigation,'' FAO (Food and Agriculture Organization of the United
  Nations, 2010), http://www.fao.org/3/i1881e/i1881e00.pdf.} Other
notable climate-smart improvements that the FAO endorses are soil
management, pest control, and better management of harvest supply
chains.\footnote{Ibid.} \\

In recent years, a component that has been severely overlooked by policy
makers is the impact climate change has on fisheries. It is extremely
important to address fisheries as they make up the essential protein
source for 3.2 billion people across the world, especially in developing
tropical countries. Scientists say that as of right now, even if we do
not overfish or decimate our fish populations in any way, the world's
fish population could drop by as much as a quarter by the end of the
century if greenhouse gas emissions continue growing at their current
rate.\footnote{Georgina Gustin et al., ``Climate Change Threatens the
  World's Fisheries, Food Billions of People Rely On,'' InsideClimate
  News, September 29, 2019,
  https://insideclimatenews.org/news/27092019/ocean-fish-diet-climate-change-impact-food-ipcc-report-cryosphere.}
Not to mention as sea temperatures rise, they will also have much lower
oxygen levels and increase in acidification across the globe. Not only
does science suggest this will lead to fish becoming smaller in size, it
will also affect the amount of plankton in the water, meaning many fish
food sources will be at risk. Other possibilities due to climate change
include, shellfish being unable to develop shells properly, harmful
algae blooms, and populations of fish will move to cooler waters,
leaving behind the fishing communities and economies that rely on
them.\footnote{Ibid.} \\

\subsection{Conclusion}

Throughout the history of the FAO as a committee, the mitigation of
climate change has always been a balancing act. There are effective ways
in which Member States can act on dealing with the issues that come with
climate change such as soil erosion, run off, etc., but when it comes
down to dealing how we can mitigate the effects the agriculture industry
has on climate change, it's a different story. As stated above, the
agricultural industry is the second largest contributor to climate
change in the world, but it is important to keep in mind that it is one
of the biggest economic sectors for LMIC across the globe. Looking for
solutions on both sides is extremely important in order to create a
balanced plan that helps tackle the threats climate change poses as well
as being able to lower the agricultural sectors impact on the climate
that causes these threats. \\

\subsection{Committee Directive}

The directive of the Food and Agriculture Organization is to address the
issue of climate change in the agriculture sector. It is expected that
you formulate solutions by addressing the effects of climate change on
the agriculture sectors across the world that are feasible for all
members of the body. It is important to also take into account the
effects that the agricultural sector has on climate change as well and
to look for proactive solutions to help mitigate these issues. Your
solutions can build off of pre-existing policies of the FAO, in an
effort to reaffirm or push for more global involvement in the policy, or
you can design new creative ways in which to deal with these issues.
Though it is important to remember that not all members of the body will
have the same economic capabilities as others, so it is important to
create solutions with these limited financial capabilities in mind. You
can address these issues in financing policy with the implementation of
creative funding mechanisms or traditionally. \\

\subsection{Questions to Consider}

\begin{itemize}
\item
  
  How does climate change affect agriculture in the Member State that I
  am representing?
  
\item
  
  Does my solution create issues for Member States that are susceptible
  to food insecurity?
  
\item
  
  What are some innovative practices or systems that can be expanded
  upon in order to help create a solution?
  
\end{itemize}

\end{document}
