\documentclass[10pt, letterpaper]{article}
        \usepackage[utf8]{inputenc}
        \usepackage[margin=1in]{geometry}
        \usepackage{fancyhdr}
        \usepackage{titling}
        \usepackage{enumitem}
        \usepackage{mathtools}
        \usepackage{amssymb}
        \usepackage{xfrac}
        \usepackage{booktabs}
        \usepackage{graphicx}
        \usepackage{wrapfig, blindtext}
        \usepackage{hyperref}
        \usepackage{enumerate}
        \usepackage{multicol}
        
        

        
        \setlength{\parindent}{0pt}

\title{Journal 2}
        \author{Sudhan Chitgopkar}
        \date{\today}
        
        % headers -- no need to change
        \pagestyle{fancy}
        \fancyhf{}
        \lhead{GA 1}
        \chead{UGAMUNC XXVII}
        \rhead{\thedate}

\begin{document}
Dear Delegates,\\

Welcome to the Disarmament and International Security
Committee! My name is Maryam Shokry, and I am thrilled to serve as your
chair for the UGA's 27\textsuperscript{th} Model UN Conference, General
Assembly 1 Disarmament and International Security Council. I am from
Marietta, Georgia, and I am a Senior in the Terry College of Business
and School of Public and International Affairs, pursuing degrees in
Economics and International Affairs, with a concentration in Security
and a minor in Statistics. Outside of my academic work, I intern at the
Western Circuit Public Defender's Office, and last summer I studied and
volunteered abroad in Stellenbosch, South Africa. In the little free
time I have, I enjoy spending time with my friends, cooking, and
watching Law and Order. \\

My name is Kartik Khanna and I will be your co-chair! I was born and
raised in the South and graduated from Northview High School in Johns
Creek, GA in 2017. At Northview, I was Secretary-General of my high
school MUN team and competed at conferences ranging from UGAMUNC to BMUN
(UC Berkeley) to NHSMUN! At UGA, I am currently a sophomore studying
Finance and hope to work in investment banking or venture capital after
graduation. In addition to UGA Model UN, I am involved with the Apollo
Society (a Terry College of Business finance organization), the Corsair
Society, and our Student Managed Investment Fund, and am a budding
photographer! In my free time, you can most likely catch me cheering on
the Dawgs or binge-watching Netflix! \\

As a delegate in our committee, we expect that you will compete
to the best of your ability and prepare adequately. With this said, we
also would like to address that this committee will at times discuss
politically sensitive topics. Thus, we expect that you will compete with
the highest level of professionalism and debate responsibly.
Additionally, as delegates of your country we expect that the scope of
your position papers and your proposed strategies in debate are in line
with the views of your country. Delegates should consider the history,
politics, culture, and the demographics of the country which they
represent. Even if you do not personally agree with these views, the
work of DISEC is meant to put forth resolutions for each country and
address problems facing the international community, therefore it is
imperative that each country is represented in their true form.
Furthermore, I would like to emphasize that every country is vital in
voicing their opinion during debate, in drafting resolutions, and in
cooperating with other nations to bring solutions to the table. \\

At the beginning of this committee we will review parliamentary
procedure and the expectations, however, my co-chair and I urge you to
review the UGAMUNC rules and procedure on our website. Please contact me
with any questions you might have at my email provided below. Finally,
please submit your completed position papers to me and CC Kartik by
11:59 PM February 1\textsuperscript{st}, as we want you to be prepared
well before the conference. \\

Best of Luck,\\
Maryam Shokry and Kartik Khanna
\\\href{mailto:mas74556@uga.edu}{\nolinkurl{mas74556@uga.edu}},
\href{mailto:kartik.khanna@uga.edu}{\texttt{kartik.khanna@uga.edu}}

\newpage
\tableofcontents
\newpage
\section{Background}

The United Nations General Assembly was established as one of the main
governing bodies of the UN with the original UN charter from
1945.\footnote{Charter of the United Nations, 1945, Art. 7} The General
Assembly and other UN bodies were given the power to create subsidiary
organs as needed, and thus were born the six main committees of the UN,
the first of which originally focused more on political and security
issues, but in the 70's became what it is today, when the committee was
recommended by the General Assembly to, "devote itself primarily to
problems of peace, security and disarmament."\footnote{"United Nations,
  Main Body, Main Organs, General Assembly." United Nations. \\

  http://www.un.org/en/ga/about/ropga/anx4.shtml\#a.} Today, the First
Committee on Disarmament and International Security provides a space for
states to discuss their various positions on disarmament related matters
and work as a team to create compromises or enact resolutions that
create tools to better understand and approach disarmament and
international security-related issues.\footnote{First Committee of the
  UN General Assembly.
  http://www.reachingcriticalwill.org/disarmament-fora/unga.} The
committee creates the opportunity for states to build consensus on
issues and tries to create a paradigm shift in which states no longer
ensure "security" for themselves through the size of their arsenals, but
rather, negotiate cooperative security arrangements that lower spending
on weapons, reduce arms production, trade, and stockpiles, along with
increasing global security.\footnote{Ibid.} \\

More recently, the committee's work and goals on disarmament have
included a review conference on the Nuclear Non-Proliferation Treaty, a
Review Conference of the Programme of Action to combat illicit
trafficking in small arms and light weapons, along with resolutions that
established a nuclear-weapon-free-zone in Africa, along with resolutions
that created similar zones in Central Asia and the southern
hemisphere.\footnote{"FEATURE: The UN General Assembly's First Committee
  - Disarmament and International Security Issues \textbar{} UN News."
  United Nations.
  https://news.un.org/en/story/2012/12/429112-feature-un-general-assemblys-first-committee-disarmament-and-international.}
The body has also been responsible for the immediate resumption of
negotiations on a treaty that bans the production of fissile (nuclear)
materials for military purposes, along with demanding that international
legal instruments be established to guarantee security for non-nuclear
states.\footnote{Ibid.} \\

On the powers of the committee, DISEC works closely with the United
Nations Disarmament Commission, and the Geneva-based Conference of
Disarmament.\footnote{"United Nations, Main Body, Main Organs, General
  Assembly." United Nations. Accessed October 26, 2018.
  http://www.un.org/en/ga/first/.} The committee in its modern forms
seeks out solutions on challenges and threats to global peace in the
international community and seeks solutions to problems in the
international security regime.\footnote{Ibid.} \\

The reality of the modern debate in DISEC is often static discussion
caused by states' limited knowledge of other state's perspective,
leading to a committee whose members have become entrenched in their own
positions and opposition to resolutions that would otherwise demonstrate
consensus on disarmament-related issues.\footnote{First Committee of the
  UN General Assembly. Accessed October 26, 2018.
  http://www.reachingcriticalwill.org/disarmament-fora/unga.} \\

\newpage
\section{Topic A: Autonomous Weapons Systems}

\subsection{Introduction}

There have been a few fundamental shifts in the
landscape of warfare. The first was the development of gunpowder, which
was followed by the eventual development of nuclear weapons. Both
technologies shifted the nature of warfare. Arguably, the next warfare
revolution will be Lethal Autonomous Weapons (AWS), which have the
ability to kill without immediate human control. Controversy on this
topic has arisen in recent years from many countries and researchers.
Many robotics and even tech firms have publicly stated that these
weapons must be banned, and even the UN Secretary General has pushed for
a full autonomous weapons ban. With the persistence of advancement in
technology and the perpetual drive for security, this issue will likely
continue to be considered in the UN discourse. Questions have arisen on
how to develop regulation within this technological space. It is the job
of the Disarmament and International Security Committee to decide how
nations will approach this ethical and political issue, as well as
debate an international legal solution to the future of Autonomous
Weapons Systems.

\subsection{History}

Two opposing viewpoints underlie this debate. One side argues that
autonomous weapons are less susceptible to human-error and communication
technology issues. Additionally, since these weapons are more precise,
they will reduce collateral damage, making war potentially more humane.
However, the other side points to serious flaws in these assumptions.
One issue is that the weapon can be programmed to search for particular
characteristics such as age or ethnicity which could result in
disproportionate selective targeting. Such implications result in a
widespread worry over a new arms race and overall global instability. \\

Currently, it is common to hear about the use of semi-autonomous
technology in war. Technology such as aircraft autopilot and missile
defense systems have not raised much controversy because of their
non-lethal components.\footnote{``Lethal Autonomous Weapons Systems.''
  futureoflife.org. Future of Life Institute, August 12, 2020.
  https://futureoflife.org/lethal-autonomous-weapons-systems/?cn-reloaded=1.}
Artificial intelligence (AI) in defense is very much a current issue, as
this is new technology and it is uncertain the direction it will take.
It is also a gradual process for this technology to develop. We have
seen many components of weapons incrementally replaced by autonomous
technology, but this is accelerating rapidly. There is an estimation
that by 2025 global military spending will invest up to \$18 Billion in
AWS and AI weapons systems.\footnote{Haner, J. and Garcia, D. (2019),
  The Artificial Intelligence Arms Race: Trends and World Leaders in
  Autonomous Weapons Development. Glob Policy, 10: 331-337.
  doi:10.1111/1758-5899.12713} A significant portion of this spending is
coming from the biggest militaries in the world including the United
States, China, and Russia.\footnote{Ibid.} However, not only is there a
concern about a global arms race, but also the proliferation of these
weapons outside of state governments. As with other new technologies
when development happens, costs decrease which allows for greater
production. This is exemplified by a similar development with drone
technology. With diminishing costs, drones have been acquired by smaller
militaries, and non-state actors. For instance, groups such as ISIS and
Boko Haram have used weaponized drones in recent years, despite not
having near the military budget or capabilities as many countries. This
issue translates to the many worries researchers and international
organizations have regarding the proliferation of Lethal Autonomous
Weapons. \\

With fully autonomous weapons peace might be disrupted on a more
significant level than with nuclear weapons or drones. This partly
comes from the nature of the technology. The appeal of AWS is primarily
its precision, deadly impact, and the attacker oftentimes being
unidentifiable because of their distance from the weapon. This results
in the possibility for greater discriminatory practices and military
overstep. These characteristics of AWS make the need for public
accountability even important. \\

\subsection{Past UN Action}

The United Nations has played a significant role in hosting
discussions on developing weapons technologies. The area of Autonomous
Weapons was raised starting in 2012 at human rights forums through the
UN. There are a few past UN treaties which have the ability to guide the
development and debate on AWS. The first is the 1949 Geneva conventions,
which fundamentally outline humanitarian priorities for wartime. The
main provision presented as international law is the protection of
civilian persons from war, and the humane treatment of medical
personnel, prisoners of war, and other people during wartime.\footnote{``Geneva
  Convention Relative to the Protection of Civilian Persons in Time of
  War of 12 August 1949.'' un.org. United Nations, n.d.
  https://www.un.org/en/genocideprevention/documents/atrocity-crimes/Doc.33\_GC-IV-EN.pdf.}
Another aspect of this treaty is that it outlines anti-discrimination
provisions during wartime. Each of these are applicable to the AWS
debate as some main concerns of the use of this technology may go
against the generally accepted international principals previously
negotiated within the UN. \\

Furthermore, the more recent and more applicable convention is the
\emph{Convention on Prohibitions or Restrictions on the Use of Certain
Conventional Weapons Which May Be Deemed to Be Excessively Injurious or
to Have Indiscriminate Effects} (CCW).\footnote{``Convention on Certain
  Conventional Weapons -- UNODA.'' un.org. United Nations, December
  2014. https://www.un.org/disarmament/publications/more/ccw/.} This
document was negotiated under UN oversight in 1979-1980 and has as
recently as 2018 been set by the UN as an applicable forum for the
debate on Autonomous Weapons. As it pertains to AWS, this treaty sets
the groundwork for ``excessively injurious or indiscriminate''
conventional weapons to be regulated by international law. Original
protocols within this convention included restriction on land mines,
blinding laser weapons, and incendiary weapons.\footnote{Ibid.} Some of
these provisions have been added in the years since it was first
negotiated. The convention has been continuously held as an
international forum, where debate regarding Autonomous Weapons Systems
began between 2014 and 2016. However, these have not been focused on
regulation or prohibition, but rather on the implications of the
technology. More substantive groups were developed in 2017 and 2018 to
engage multinational technology experts, governmental experts, human
rights NGOs, and various think tanks such as the Stockholm International
Peace Research Institute.\footnote{Ibid.} \\

Within the CCW platform, the UN has outlined a number of principles for
the international development of autonomous weapons. These include
accountability for use of force, as with many other weapons, weapons
reviews, and the development of cybersecurity and proliferation
safeguards.\footnote{Gill, Amandeep Singh. ``The Role of the United
  Nations in Addressing Emerging Technologies in the Area of Lethal
  Autonomous Weapons Systems.'' un.org. United Nations. Accessed October
  31, 2020.
  https://www.un.org/en/un-chronicle/role-united-nations-addressing-emerging-technologies-area-lethal-autonomous-weapons.}
Additionally, the UN has indicated that these weapons are largely being
created by the private industry rather than governmental institutions.
Therefore, there is a need for multinational forums and experts to
cooperate with all stakeholders and involve the private industry sector
in the development of international policies on AWS. \\

\subsection{International Development}

Capabilities in Autonomous Weapons Systems are a combination of
private industry and state led development. However, it is global
militaries that are ultimately the consumers of AWS technology.
According to defense spending numbers as of 2019, the five largest
competitors for these systems are the United States, China, Russia,
South Korea, and the European Union as a combined GDP entity.\footnote{Stauffer,
  Brian. ``Stopping Killer Robots.'' hrw.org. Human Rights Watch, August
  10, 2020.
  https://www.hrw.org/report/2020/08/10/stopping-killer-robots/country-positions-banning-fully-autonomous-weapons-and.}
It is clear that these countries align with historic global military
competition, such as in the Cold war period, and they also have the
largest global military budgets. These facts are the primary cause for
the concern of a global arms race arising out of AWS development.
However, it is the hope of the United Nations that international norms
and treaties such as the CCW will be a regulatory factor as AWS
continues to be developed. \\

 As discussed earlier, AI and AWS technology comes with a high
development cost, however, this is changing as production expands and
becomes less costly. Furthermore, as it currently stands, most countries
with established or developing capacity in AWS are parties to the CCW.
This is a crucial element of the ability for international forums and
regulation to proceed with those involved in technology development.
Additional countries with these capabilities include EU countries such
as France and Germany, Israel, Canada, and Japan.\footnote{Ibid.}
Smaller countries which are allied with larger militaries also have the
potential for access to these emerging technologies. These countries,
typically defined as having a smaller GDP, also have a vested interest
in this debate as global warfare and asymmetric power rising from
powerful technology will potentially impact their sovereignty and
security. \\

 In addition to development within individual countries, the
private industry for AWS is multinational. As previously mentioned,
states have an interest in working alongside industry stakeholders to
maintain an accountability in the international community. Some private
firms maintain that autonomous weapons are superior to traditional
weapons because they are not reliant on sometimes unpredictable human
control. Furthermore, since the weapons are more targeted, casualties
should be reduced as the weapon could eliminate an individual or
building without collateral damage. Stigma in the international
community by those who fear this technology and so called ``killer
robots'' may have an adverse effect. This fear might cause the private
industry to retract from the public eye and continue their research in
secrecy.\footnote{Ibid.} As many international institutions and states
value public oversight and cooperation, this could be a detrimental
development within the autonomous weapons technology space. \\

 As Autonomous Weapons Systems continue to develop, it is in the
interest of the Disarmament and International Security Council to be at
the forefront of this debate. Many controversies still remain on the
table, such as whether or not this technology should even be allowed to
be developed. This committee will engage in debate regarding the
potential prohibition and regulation of this area in weapons technology.
Furthermore, debate will surround the boundaries for permissible use of
force, and how sovereignty should be considered with respect to
Autonomous Weapons Systems. \\

\subsection{Vocabulary}

\begin{enumerate}
\def\labelenumi{\arabic{enumi}.}
\item
  Autonomous: Something that has the ability to operate without outside
  control or having the right to self-government.\footnote{``Autonomous.''
    Merriam-Webster. Merriam-Webster. Accessed October 31, 2020.
    https://www.merriam-webster.com/dictionary/autonomous.}
\item
  Non-State Actor: An organization, company or individual which has the
  ability to wield with political and social power but is not directly
  part of the government.\footnote{``Non-State Actors.'' escr-net.org.
    ESCR. Accessed October 31, 2020.
    https://www.escr-net.org/resources/non-state-actors.}
\item
  CCW: This is the Convention on Certain Conventional Weapons. It
  regulates weapons that cause excessive injury or have indiscriminate
  effects.\footnote{Ibid}
\item
  GDP: The comprehensive value of goods and services produced within a
  country's borders.\footnote{``Gross Domestic Product.'' Gross Domestic
    Product \textbar{} U.S. Bureau of Economic Analysis (BEA). Accessed
    October 31, 2020.
    https://www.bea.gov/data/gdp/gross-domestic-product.}
\end{enumerate}

\subsection{Questions to Consider}

\begin{enumerate}
\def\labelenumi{\arabic{enumi}.}
\item
  How should the international community respond to calls for both
  oversight or the overall banning of Lethal Autonomous Weapons Systems?
  Particularly consider incentives, costs, and impact.
\item
  What are both positive and negative impacts of the development of AWS?
\item
  What levels of regulation of AWS should be considered, and what are
  the costs and incentives for regulation/deregulation?
\item
  What are proposed solutions to the proliferation of AWS to non-state
  actors?
\item
  How will the UN incorporate and address industry stakeholders who are
  leading the technological development of AWS?
\end{enumerate}

\subsection{Further Reading}

\begin{enumerate}
\def\labelenumi{\arabic{enumi}.}
\item
  \emph{Group of Governmental Experts of the High Contracting Parties to
  the Convention on Prohibitions or Restrictions on the Use of Certain
  Conventional Weapons Which May Be Deemed to Be Excessively Injurious
  or to Have Indiscriminate Effects} http://undocs.org/ccw/gge.1/2017/WP.1
\end{enumerate}


\begin{enumerate}
\def\labelenumi{\arabic{enumi}.}
\setcounter{enumi}{1}
\item
  \emph{So Just What Is a Killer Robot?: Detailing the Ongoing Debate
  around Defining Lethal Autonomous Weapon Systems} https://www.whs.mil/News/News-Display/Article/2210967/so-just-what-is-a-killer-robot-detailing-the-ongoing-debate-around-defining-let/
\end{enumerate}



\begin{enumerate}
\def\labelenumi{\arabic{enumi}.}
\setcounter{enumi}{2}
\item
  \emph{Autonomous Weapons Systems and the Laws of War,}\\ https://www.armscontrol.org/act/2019-03/features/autonomous-weapons-systems-laws-war

\end{enumerate}


\newpage
\section{Topic B: Military Tensions and Tensions in the South China
Sea}

\subsection{Introduction}

 Since the early 1950s, China began a campaign to expand its
territory and influence worldwide, from issuing debt in the form of
``foreign aid'' to default-prone developing countries to funding foreign
infrastructure projects ``free of cost''. However, nowhere has this
campaign been more apparent than in the South China Sea, a strait
between the Indochinese Peninsula in the North, and Indonesia, Malaysia,
and Brunei in the South. While the majority of this strait is considered
international waters by the UN and other bodies, China claims over 90\%
of the strait is Chinese territory --- and often backs up any claims
with subsequent military defensive and offensive measures. NGOs estimate
that China's illegitimate South China Sea presence vastly expands the
country's implied regional power and increases military operational
ranges by up to 1,000 kilometers.\footnote{``South China Sea Dispute:
  China's Pursuit of Resources `Unlawful', Says US - BBC News.''
  Accessed 30 October 2020.
  https://www.bbc.com/news/world-us-canada-53397673.} However, China is
not the only country to pursue aggressive territorial actions, as
countries such as Taiwan, Philippines, Japan, and Malaysia lay their own
respective regional claims for similar reasons. One must note, however,
that the reasons for these territorial disputes are both strategic and
nationalistic, especially when noting the lasting effects of Japan's
defeat in WWII and subsequent Cold War geopolitical events. In the
current day, the consequences of militarizing supposed ``no-man's land''
islands and China's land reclamation efforts in the Strait prove
concerning, both geo-politically and economically (noting the \$3.37T
USD in trade passing through the strait yearly and the massive energy
reserves the strait holds). \\

\subsection{History}

 The South China Sea has been a volatile region throughout
history, partly due to tense intra-regional relationships, especially
between Japan and China. Regional volatility first started after Qing
China's defeat in the First Sino-Japanese War, which enabled Japan's
expansion into the strait, and then again after Japan's defeat in WWII
and the subsequent territorial ``tug-of-war''. However, in historical
times, territorial claims were never a high priority for any regional
power, with most action tempered by other, more pressing conflicts
requiring attention. The crisis truly gained momentum in the
early-2010s, with more-aggressive territorial actions and increased
conflicts.\footnote{``South China Sea.'' Accessed 30 October 2020.
  https://www.lowyinstitute.org/issues/south-china-sea.} \\

 Many stakeholders base their regional claims through partisan,
unverifiable anecdotes claiming national presence on the Strait's
islands since antiquity, citing archaeological evidence pointing towards
fishing activity or shelter construction on islands in question. China's
Communist Party, for instance, actively subsidizes archeological digs in
the region, hoping to substantiate its claims. Post-WWII, Chiang
Kai-shek's China laid claims to islands in question, seeing a lapse in
Japanese regional authority and low repercussions. However, after Mao
Zedong defeated and exiled Kai-shek and his followers to Taiwan, the
Strait's islands were abandoned and laid unbothered, especially noting
other, more pressing matters operating in parallel --- the Vietnam War.
This period of relative calm was interrupted in 1955, when both China
and Taiwan laid claims to S. China Sea islands, with a Philippine
citizen claiming the whole of Spratly Island for himself. However, this
``Scramble for the South China Sea'' truly gained momentum in the 1970s
when oil was potentially discovered in the region, keeping in mind the
geopolitical repercussions of the 1973 OPEC oil crisis. In this time
period, the Philippines first occupied the Strait's territory, with
China following soon after with a seaborne invasion of several islands.
For instance, in the \textbf{Battle of the Paracel Islands}, China won
several islands from South Vietnam, but killed several Vietnamese
soldiers in the process.\footnote{``Territorial Disputes in the South
  China Sea.'' Global Conflict Tracker. Accessed 30 October 2020.
  https://cfr.org/global-conflict-tracker/conflict/territorial-disputes-south-china-sea.}
However, even this sequence of events soon calmed until 1988, when China
invaded the Spratly Islands, thus spurring another round of territorial
occupations by many regional parties. \\

\subsection{UN and International Response}

 In 2002, peace seemed more-and-more realistic as ASEAN \& China
signed the Declaration on the Conduct of Parties in the South China Sea,
which detailed a framework to eventually develop a Code of Conduct for
the strait. Through this declaration, all signatories agreed to
``exercise self-restraint in the conduct of activities that would
complicate or escalate disputes and affect peace and stability
including, among others, refraining from action of inhabiting on the
presently uninhabited islands, reefs, shoals, cays, and other features
and to handle their differences in a constructive manner''. This
declaration was largely successful in tempering regional conflict for
about 5 years, with all parties refraining from claiming additional
features.\footnote{``U.S. Position on Maritime Claims in the South China
  Sea.'' United States Department of State. Accessed 30 October 2020.
  https://www.state.gov/u-s-position-on-maritime-claims-in-the-south-china-sea/.}
However, the Declaration showed signs of tension, with signatories
``testing the waters'', so to speak, through demarches and notes
verbales which aimed to both outline additional potential claims and
voice discontent. Famously, Malaysia \& Vietnam's joint submission to
the Commission on the Limits of the Continental Shelf outlined these
aforementioned claims. China's ``nine-dash line'', which drew a line
encompassing most of the strait and all of its land features while
engaging in ``strategic ambiguity'' and refusing to explain what the
line signifies, illustrates the indirect ways parties began to flex
their muscles. \\

 However, the current problem in the region can be traced to
increasing aggression and the region's newly reaffirmed strategic
importance. As previously mentioned, over \$3.3T USD passes through the
strait yearly and 11 billion barrels of oil are present in the region.
Keeping this in mind, in 2012, China took control of Philippines'
Scarborough Shoal under false pretenses. After a two-month long
standoff, both China and Philippines agreed to relinquish the shoal's
control, but only the Philippines honored the accord. Later, the
Philippines asked the UNCLOS (UN Convention on the Law of the Sea) to
arbitrate. However, China refuses to participate in the arbitration,
arguing that maritime disputes cannot be solved before territorial ones
(territorial disputes fall outside the UNCLOS charter) In 2014, a
Chinese state-owned oil company moved an oil rig into Vietnam-claimed
territorial waters until violent confrontation between Vietnam and China
forced the company to withdraw its rig one month ahead of schedule.
Moreover, recent Chinese land reclamation projects aiming to
unilaterally cement the country's regional presence have exponentially
escalated tensions, with even the United States sending its military to
the region to ``preserve the strait's international
status''.\textsuperscript{12} \\

While this problem seems regional, the strait is a vital shipping route
for international commerce, with 30\% of global maritime oil trade
passing through the region.\textsuperscript{13} Moreover, increasing
military presence threatens regional stability and sets dangerous
territorial presence for the rest of the international community. Noting
parties' disregard for international treaties and the UN's historical
ineffectiveness when arbitrating this situation, successful solutions
must effectively hold all regional parties accountable while protecting
global interests.

\newpage
\section{Topic C: Privatization of War and Private Military Security
Companies}

\subsection{Introduction}

Traditionally held beliefs have dictated that providing security for a
state is ultimately the responsibility of that state government. This
stems both from the perspective that to maintain power and sovereignty a
state needs to provide itself security, as well as that it has a
responsibility to provide security for its citizens. These ideals are
being challenged with the expansion of Private Military Security
Companies (PMSC). These companies have both been engaged in the
development of weapons, but also in providing tactical military support
and personnel for hire. Consequently, these multinational corporations
lead a security industry outside of the control of state governments,
and often provide support for multiple states at the same
time.\footnote{``UN Expert Group Cites Need for Global Instrument
  Regulating Private Security Companies \textbar{} \textbar{} UN News.''
  news.un.org. United Nations, November 4, 2013.
  https://news.un.org/en/story/2013/11/454582-un-expert-group-cites-need-global-instrument-regulating-private-security.} \\

Many concerns have been raised regarding the operation of these security
companies including the potential for the violation of human rights.
Since they are not accountable to the public interest, but rather their
customers and stakeholders the incentive structure for these companies
differs greatly from that of a state military. Although some argue this
is an adverse effect, others argue that PSMCs promote innovation in the
security space and provide state governments the ability to defend their
civilians without sacrificing as many of their state military personnel.
These two points allow for political gains in many state governments,
and innovation is regarded as a general social benefit. It is the job of
the Disarmament and International Security Committee to decide how
nations will approach regulation in this space, as well as debate
implications of these Private Security Companies in the international
realm. \\

\subsection{History}

For profit militaries, or historically known as mercenaries have been
around for ages. However, as economic systems shifted, so did the
industry of private militaries. Rather than a group of soldiers for
hire, today's for-profit military industry is made up of many
multinational corporations. Beyond the economic models used for this
industry the shape of the demand for military power changed
internationally. \\

The last significant international conflicts were World War II and
followed by the Cold War a few decades later. After the devastation from
the new weapons developed during this time period was clear, there was a
diminished need for soldiers globally.\footnote{Singh, Ana. ``Soldiers
  of Fortune: the Rise of Private Military Companies and Their
  Consequences on America's Wars.'' Berkeley Political Review, October
  25, 2017.
  https://bpr.berkeley.edu/2017/10/25/soldiers-of-fortune-the-rise-of-private-military-companies-and-their-consequences-on-americas-wars/.}
This led to the decline in the sizes of the standing armies of the
historically largest militaries. Additionally, the structure of war has
consistently shifted, but even more so after the Cold War. Conflicts we
see now are more regional and fought more commonly as proxy wars, rather
than major superpowers directly fighting each other. These factors led
to the expansion of the demand and market of private military
contractors. Rather than internally producing military ``goods'', it
became more efficient to outsource this production, of both weapons and
security forces. After these general dynamics in warfare, the most
notable shift in the use of military contractors globally was their use
in Iraq and Afghanistan by the United States starting in 2009. In fact,
during this conflict period military contractors began to outnumber the
total U.S. forces without lowering the overall number of troops
stationed in the Middle East operating for the U.S. A similar dynamic
has historically been taking place in Latin American also at the
direction of the United States Government. Furthermore, some of the U.S.
based private firms are simultaneously selling weapons to other
militaries or providing training services, such as with countries like
Australia or the Democratic Republic of the Congo (DRC). This dynamic
undoubtedly leads to debate on the international security implications
of this industry. \\

Many other States and regions have adopted the use of PMSC. Another
example is the expansion in the last decade of PMSCs operating in the
Indian Ocean and the Horn of Africa to protect merchants against
piracy.\footnote{Brown, James. ``Pirates and Privateers: Managing the
  Indian Ocean's Private Security Boom.'' globalpolicy.org. Global
  Policy Forum, September 12, 2012.
  https://www.globalpolicy.org/nations-a-states/private-military-a-security-companies/51915-pirates-and-privateers-managing-the-indian-oceans-private-security-boom.html?itemid=id.}
This region has many smaller insurgency groups which fund themselves
partially through piracy. As mentioned earlier, since global standing
armies have become smaller, regional governments in this area have had
to outsource their economic security interests to PMSCs. One report
claims that Singapore has arguably the most sophisticated private navy
as their economy relies heavily on overseas trade. Although the largest
militaries are often thought of as at the forefront of military spending
and development, each country has a vested interest in overall global
security. With the rise of PMSCs, even the smallest state could both
benefit or be severely harmed by the expansion of these companies. The
last few decades have provided evidence of the scale of operations of
PMSCs, and their presence within each continent. \\

\subsection{Past UN Action/International Involvement}

Ironically, the United Nations itself utilizes private militaries,
deploying these security forces in both humanitarian and peacekeeping
missions. With the rising use of private militaries by the UN
(especially in the last decade), many are concerned about the inherent
transparency and accountability gaps that rise as a result. For
instance, throughout the 2010's, the UNHCR's Mercenary Working Group
evaluated the increasing risk of human rights violations (and lack of
accountability) from the deployment of privatized military
forces.\footnote{Royal Norwegian Naval Academy, and Åse Gilje Østensen.
  UN Use of Private Military and Security Companies. November 2011.} \\

However, on the other hand, independent analysts observe how private
military forces do not serve on front-line missions, rather providing
middle-and-background operations, which significantly reduces risk of
over-privatization of UN missions. Moreover, the command structure of
many private military forces (PMF) is not agile enough to enable rapid
deployment at scale, which reduces the possibility of PMFs replacing
current, member-state provided forces.\footnote{Lynch, Colum. ``U.N.
  Embraces Private Military Contractors.'' Foreign Policy. Accessed 30
  October 2020.
  https://foreignpolicy.com/2010/01/17/u-n-embraces-private-military-contractors/.}
Just as international member-states, the UN has embraced private
security forces to bolster UN deployments \& secure UN sites worldwide,
especially the Middle East.\footnote{``Use of Private Military and
  Security Companies by the United Nations.'' PeaceWomen, 3 February
  2015.
  https://www.peacewomen.org/e-news/article/use-private-military-and-security-companies-united-nations.}
Thus, any solution generated in committee \textbf{must} tackle the
accountability gaps present in worldwide PMF deployment while bridging
the gap that PMFs currently fill. \\

\subsection{Vocabulary}

\begin{enumerate}
\def\labelenumi{\arabic{enumi}.}
\item
  PMSC: This the abbreviation for a private military security company.
  These are corporations whose services are hired by governments to act
  in a military capacity.
\item
  Outsourcing: Hiring services outside of your internal company or
  organization to provide goods and services that have previously been
  produced within the organization.\footnote{Twin, Alexandra. ``Why
    Companies Use Outsourcing.'' Investopedia. Investopedia, September
    16, 2020. https://www.investopedia.com/terms/o/outsourcing.asp.}
\end{enumerate}

\subsection{Questions to Consider}

\begin{enumerate}
\def\labelenumi{\arabic{enumi}.}
\item
  How should the international community respond to calls for oversight
  of Private Military Contractors? Particularly consider incentive
  structures and costs.
\item
  What are both positive and negative impacts of PMSCs?
\item
  What levels of regulation should be considered, and what are the costs
  and incentives for regulation/deregulation?
\item
  How will the UN incorporate and address industry stakeholders who are
  leading the development of PMSCs?
\end{enumerate}

\subsection{Further Reading}

\begin{enumerate}
\def\labelenumi{\arabic{enumi}.}
\item
  \emph{30 Most Powerful Private Security Companies in the World}
  https://www.securitydegreehub.com/most-powerful-private-security-companies-in-the-world/
\item
  \emph{Private Military and Security Companies in Somalia Need
  Regulation, Says UN Expert Group}\\
  https://www.globalpolicy.org/nations-a-states/private-military-a-security-companies/50192-countries-in-which-pmscs-operate.html
\item
  \emph{Use of military contractors shrouds true costs of war.
  Washington wants it that way, study says.} https://www.washingtonpost.com/national-security/2020/06/30/military-contractor-study/
\end{enumerate}



\end{document}
